\documentclass{article}
\usepackage[utf8]{inputenc}
\usepackage{kotex}
\usepackage{lmodern}
\usepackage{cite}

\author{지준섭}
\title{News Editorials Entailment Corpus Proposal}

\begin{document}

\maketitle
이전에 STARLAB에서 제작한 Corpus\cite{credcorpus}의 확장으로,
News Editorials 도메인에서의 Textual Entailment Corpus를 계획합니다.

\section{필요성}
기존에 Textual Entailment Corpus로서 RTE, SNLI\cite{snli:emnlp2015} 등이 존재하고,
텍스트 장르의 다양성 확보를 위해 multiNLI\cite{multinli}가 기획되었으나,
News Editorials 도메인을 포함한 일정 규모 이상의 코퍼스는 존재하지 않습니다.
Textual Entailment Problem의 경우 도메인에 대해 일정 이상의 의존도를 가지고 있기에,
해당 도메인의 코퍼스를 구축할 수 있다면 Textual Entailment 관련 연구에
많은 도움이 될 것으로 생각됩니다.
또한, 현재 진행하는 Credibility Prediction Using Relations with Sentences 연구의
첫 번째 단계(Sentence Selection)에도 큰 도움이 될 수 있을 것으로 판단됩니다.

\section{목적}
기존의 코퍼스의 텍스트(영문 사설 뉴스)를 기반으로, 임의의 두 문장 간의 관계를
Entailment/Neutral/Contradiction 세 가지로 나누어 표현한 코퍼스를 구축합니다.
학습을 통해 의미있는 경향성을 찾아낼 수 있는 것을 목표로 하며,
세 가지 분류에 대해 문장의 쌍들이 비교적 고르게 분포하는 것을 지향합니다.

\section{의의}
코퍼스의 제작으로 도메인에 구애받지 않는,
일반적인 Textual Entailment 모델의 구축에 도움이 될 수 있을 것입니다.
또한, 신뢰도 분포 연구의 경우 News Editorials 분야의 데이터가 중요하므로
코퍼스를 만듦으로써 Textual Entailment와 신뢰도 분포 사이의 관계를 밝히는 연구에
진전이 있을 것이라고 생각됩니다.

연구 문제를 해결함으로써
학문의 이론적 또는 방법론적 발전에 기여할 수 있는 것이
무엇이며 그 중요성은 어떠한지에 대해 밝히며,
실제적인 측면에서 거둘 수 있는 기여도도 함께 기술합니다.

\section{이론적 배경}
\begin{enumerate}
  \item \textbf{SNLI}: 
    Amazon Mechanical Turk를 활용하여 대규모의 데이터셋을 구축하였습니다.
    Entailment / Neutral / Contradiction의 세 가지 라벨을 제안하였고,
    각각의 라벨에 데이터가 고르게 분포하도록 조정되었습니다.
  \item \textbf{multiNLI}:
    기본적으로 SNLI의 방식을 따르되 visual scene에 대한 묘사로 장르가 제한된 것을
    극복하기 위해 제안되었습니다. 10개의 장르
    (fiction / government / slate / telephone / travel / 9.11 / face-to-face / letters / oup / verbatim)
    에 대해 코퍼스를 구축하였으며, 이 중 5개의 장르는 train set에 포함되지 않는 mismatched set으로 분류하여
    domain에 대한 generality를 확보하기 위해 노력하였습니다.
  \item \textbf{STARLAB Corpus}:
    News Editorials 도메인에 대하여 
\end{enumerate}

연구주제에 관한 기존의 원리나 법칙,
이론들을 종합하여 체계적으로 제시합니다.
선행연구의 결과를 기초로 이들을 비판하고, 자신의 연구와
이러한 이론이나 법칙들과의 논리적 관계 그리고 자기의 연구와
선행연구와의 논리적 관계를 체계적으로 밝힙니다.

이러한 관계에서 자신의 연구 문제를 해결하는데 기초가 되는
이론과 전제를 분명히 하며, 역사적 연구의 경우에는
연구자 자신이 다루려는 역사적 사실이나 사건을 어떤 관점이나
사관에서 보고자 하는지와 그러한 사관을 택한 이유,
그리고 다른 사관을 택했을 때 이들 사건이나 사실에 대한
해석이 어떻게 달라질 수 있는지에 대해 밝혀야 합니다.

문학 연구의 경우에는 채택하고 있는 비평 방식이나
접근 방식의 장단점을 밝히고, 철학적 연구의 경우에는
연구주제와 관련된 현상을 어떤 철학적 관점에서 보려는지에
대해 밝힙니다. 이런 의미에서, 연구의 이론적 배경은 연구에서
검증하고자 하는 가설 설정의 기초를 제공한다고 할 수 있습니다.

\section{가설의 설정} 
가설은 연구주제에 대한 잠정적인 대답으로,
두 변인(變人) 간의 관계를 진술한 것입니다. 
이 잠정적인 대답은 이론이나 선행연구의 결과를 기초로 설정되며,
가설은 경험적 증가를 바탕으로 그 진위(眞僞)를 확률적으로
결정할 수 있도록 진술해야 합니다.

\section{연구 방법}
 선정한 연구주제를 어떤 방식으로
해결할 것인지 그 방법을 명시합니다. 구체적으로는 누구에게
무엇을 어떻게 할 것 인지에 대해 규명하는 것으로서,
연구 절차와 연구 대상, 도구, 자료수집과 분석 및 처리 방법 등에
대해 진술합니다. 연구과제의 성격에 따라 문헌연구나
경험적 연구의 방법을 채택하게 될 것입니다.

\section{기대되는 결과}
연구의 결과가 어떻게 나타날 것으로 기대하는지에
대해 명시하고, 그에 따른 잠정적인 결론을 구상하여 기재합니다.

\bibliographystyle{plain}
\bibliography{arg-corpus-proposal}

\end{document}