\documentclass{article}
\usepackage[utf8]{inputenc}
\usepackage{kotex}
\usepackage{lmodern}
\usepackage{cite}
\usepackage{minted}

\author{20193601 지준섭}
\title{CS579 Homework\#2}

\begin{document}

\maketitle

\section{Chapter 7}
\textbf{Problem}: The formal language
$ a^{n}b^{2m}c^{2m}d^{n} $ consists of all strings of the following form:
an unbroken block of $a$s followed by an unbroken block of $b$s followed by
an unbroken block of $c$s followed by an unbroken block of $d$s,
such that the $a$ and $d$ blocks are exactly the same length,
and the $b$ and $c$ blocks are also exactly the same length
and furthermore consist of an even number of $b$s and $c$s respectively.
For example, $\epsilon$, $abbccd$, and $aabbbbccccdd$ all belong to $a^{n}b{2m}c^{2m}d^{n}$.
Write a DCG that generates this language. \\

\noindent
\textbf{Solution}:
The main idea for the problem is that,
the string satisfies the language is always surrounded by one $a$ and $d$,
or two $b$s and $c$s, if the string is not $\epsilon$.
So I defined two kind of strings: \mintinline{prolog}{asd}, and \mintinline{prolog}{bbscc}.
\begin{enumerate}
  \item \mintinline{prolog}{asd}: A sentence surrounded with a and d.
  \mintinline{prolog}{asd} can be represented as
  \mintinline{prolog}{[a], asd, [d]} again, or \mintinline{prolog}{[a], bbscc, [d]}.

  \item \mintinline{prolog}{bbsdd}: A sentence surrounded with bb and cc.
  \mintinline{prolog}{bbsdd} can be a empty string,
  or \mintinline{prolog}{[b, b], bbscc, [c, c]}.
\end{enumerate}
And here is the prolog code below.

\begin{minted}[breaklines]{prolog}
s --> [a], asd, [d].
s --> [a], bbscc, [d].
s --> bbscc.
asd --> [a], asd, [d].
asd --> [a], bbscc, [d].
bbscc --> [b, b], bbscc, [c, c].
bbscc --> [].
\end{minted}

\section{Chapter 8}
\textbf{Problem}: First, bring together all the things we have learned
about DCGs for English into one DCG.
In particular, in the text we saw how to use extra arguments
to deal with the subject/object distinction,
and in the exercises you were asked to use additional arguments
to deal with the singular/plural distinction.
Write a DCG which handles both.
Moreover, write the DCG in such a way that
it will produce parse trees, and makes use of a separate lexicon.
Once you have done this, extend the DCG so that
noun phrases can be modified by adjectives
and simple prepositional phrases
(that is, it should be able to handle noun phrases such as
“the small frightened woman on the table” or “the big fat cow under the shower”).
Then, further extend it so that the distinction between
first, second, and third person pronouns is correctly handled
(both in subject and object form). \\

\noindent
\textbf{Solution}:
Starting from the DCG we learned from the lecture, subject/object version:
\begin{minted}[breaklines]{prolog}
s --> np(subject), vp.

np(_) --> det, n.
np(X) --> pro(X).

vp --> v, np(object).
vp --> v.

det --> [the].
det --> [a].

n --> [woman].
n --> [man].

pro(subject) --> [he].
pro(subject) --> [she].
pro(object) --> [him].
pro(object) --> [her].

v --> [shoots].
\end{minted}

\noindent
Add singular/plural feature:
\begin{minted}[breaklines]{prolog}
%% SP: variable for singular / plural
s --> np(subject, SP), vp(SP).

np(_, SP) --> det(SP), n(SP).
np(X, SP) --> pro(X, SP).

vp(SP) --> v(SP), np(object, _).
vp(SP) --> v(SP).

det(singular) --> [the].
det(singular) --> [a].
det(plural) --> [the].

n(singular) --> [woman].
n(singular) --> [man].
n(plural) --> [women].
n(plural) --> [men].

pro(subject, singular) --> [he].
pro(subject, singular) --> [she].
pro(subject, plural) --> [they].
pro(object, singular) --> [him].
pro(object, singular) --> [her].
pro(object, plural) --> [them].

v(singular) --> [shoots].
v(plural) --> [shoot].
\end{minted}

\noindent
Add parse tree \& lexicon feature:
\begin{minted}[breaklines]{prolog}
s(SP, s(SP, NP, VP)) --> np(subject, SP, NP), vp(SP, VP).

np(_, SP, np(SP, DET, N)) --> det(SP, DET), n(SP, N).
np(X, SP, np(SP, N)) --> pro(X, SP, N).

vp(SP, vp(SP, V, NP)) --> v(SP, V), np(object, _, NP).
vp(SP, vp(SP, V)) --> v(SP, V).

det(SP, det(SP, Word)) --> [Word], {lex(Word, SP, det)}.
n(SP, n(SP, Word)) --> [Word], {lex(Word, SP, n)}.
v(SP, v(SP, Word)) --> [Word], {lex(Word, SP, v)}.
pro(X, SP, pro(X, SP, Word)) --> [Word], {lex(Word, X, SP, pro)}.

lex(a, singular, det).
lex(the, _, det).
lex(woman, singular, n).
lex(man, singular, n).
lex(women, plural, n).
lex(men, plural, n).
lex(shoots, singular, v).
lex(shoot, singular, v).
lex(he, subject, singular, pro).
lex(she, subject, singular, pro).
lex(they, subject, plural, pro).
lex(him, object, singular, pro).
lex(her, object, singular, pro).
lex(them, object, plural, pro).
\end{minted}

\noindent
Add adjective\&preposition feature:
\begin{minted}[breaklines]{prolog}
s(SP, s(SP, NP, VP)) --> np(subject, SP, NP), vp(SP, VP).

np(_, SP, np(SP, DET, N)) --> det(SP, DET), n(SP, N).
np(X, SP, np(SP, N)) --> pro(X, SP, N).
np(_, SP, np(DET, AP, N)) --> det(SP, DET), ap(AP), n(SP, N).
np(_, SP, np(DET, N, PP)) --> det(SP, DET), n(SP, N), pp(PP).
np(_, SP, np(DET, AP, N, PP)) --> det(SP, DET), ap(AP), n(SP, N), pp(PP).

vp(SP, vp(SP, V, NP)) --> v(SP, V), np(object, _, NP).
vp(SP, vp(SP, V)) --> v(SP, V).

ap(ap(ADJ)) --> adj(ADJ).
ap(ap(ADJ1, ADJ2)) --> adj(ADJ1), adj(ADJ2).

pp(pp(PRE, NP)) --> pre(PRE), np(object, _, NP).

det(SP, det(SP, Word)) --> [Word], {lex(Word, SP, det)}.
n(SP, n(SP, Word)) --> [Word], {lex(Word, SP, n)}.
v(SP, v(SP, Word)) --> [Word], {lex(Word, SP, v)}.
pro(X, SP, pro(X, SP, Word)) --> [Word], {lex(Word, X, SP, pro)}.
adj(adj(Word)) --> [Word], {lex(Word, adj)}.
pre(pre(Word)) --> [Word], {lex(Word, pre)}.

lex(a, singular, det).
lex(the, _, det).
lex(woman, singular, n).
lex(man, singular, n).
lex(women, plural, n).
lex(men, plural, n).
lex(table, singular, n).
lex(tables, plural, n).
lex(cow, singular, n).
lex(cows, plural, n).
lex(shower, singular, n).
lex(showers, plural, n).
lex(shoots, singular, v).
lex(shoot, singular, v).
lex(i, subject, plural, pro).
lex(you, subject, plural, pro).
lex(he, subject, singular, pro).
lex(she, subject, singular, pro).
lex(we, subject, plural, pro).
lex(they, subject, plural, pro).
lex(me, object, plural, pro).
lex(you, object, plural, pro).
lex(him, object, singular, pro).
lex(her, object, singular, pro).
lex(us, object, plural, pro).
lex(them, object, plural, pro).
lex(small, adj).
lex(frightened, adj).
lex(big, adj).
lex(fat, adj).
lex(on, pre).
lex(under, pre).
lex(in, pre).
lex(at, pre).
\end{minted}

\noindent
Here is the execution example:
\begin{minted}[breaklines]{prolog}
?- s(A, B, [the, woman, on, the, table, shoot, him], []).
A = singular,
B = s(singular, np(det(singular, the), n(singular, woman), pp(pre(on), np(singular, det(singular, the), n(singular, table)))), vp(singular, v(singular, shoot), np(singular, pro(object, singular, him)))) ;  
?- s(A, B, [the, small, frightened, woman, in, a, shower, shoot, them], []).
A = singular,
B = s(singular, np(det(singular, the), ap(adj(small), adj(frightened)), n(singular, woman), pp(pre(in), np(singular, det(singular, a), n(singular, shower)))), vp(singular, v(singular, shoot), np(plural, pro(object, plural, them)))) ;
\end{minted}
\end{document}