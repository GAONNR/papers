\documentclass[11pt]{article}
\usepackage[scale=0.75,twoside,bindingoffset=5mm,a4paper]{geometry}
\usepackage[utf8]{inputenc}
\usepackage{kotex}
\usepackage{lmodern}
\usepackage{cite}
\usepackage{minted}

\author{20193601 지준섭}
\title{CS579 Homework\#2 -- Proposal \\
\begin{large}
  Automatic Debt Management System With Messenger Conversation
\end{large}}

\begin{document}

\maketitle

Type: \textit{implemented system with documentation}

\section{Introduction}

While belonging to a group, such as circle, laboratory, or the company,
there is always a problem of financial relationship.
The problem can be about a food ate together, a product purchased instead,
or other important one.
Building a system which saves those financial records, and shows summary
would be useful.
But in that case, users must learn how to register the records.
So it would be much easier if we build a system that automatically
extracts financial information from the conversation in the messenger.

\section{Details}

For the term project, we will build a system for a particular messenger
-- \textit{Slack}.
\textit{Slack} is a widespread messenger for groups,
such as circles, laboratories, and the companies.
It provides an API to access the messages,
so people can easily build a bot when they need a specular functionality.
In this project, our system will automatically analyze the chat,
extracts financial information(e.g., debts and receivables)
between the members in the group, and upload it to the server as a transaction.
For sentence parsing, we plan to use libraries like
\textit{node-swipl}, or \textit{pyswip}
because the sentences will be crawled by \textit{node.js} or \textit{Python},
and they will be parsed by \textit{SWI-prolog}.


Also, our system will provide the summarization and the details
of the financial relationship.
The information will be provided as web page.
The web page will be served by libraries like
\textit{express.js}, or \textit{Flask}.

\section{Scenario}

\begin{enumerate}
  \item \textbf{User requests money to other people} \\
  For example, a user bought a dinner to eat with four people together,
  (i.e., 5 people ate dinner together)
  and want to request a money.
  User can say ``Dinner is 50,000 won in total.'',
  or ``Give me 10,000 won each.''.
  The system should analyze both kind of request,
  and assign a proper amount of money to the people.

  \item \textbf{User want to notice he/she has paid debt} \\
  After sending a proper money, user will say like
  ``I just sent 10,000 won to you.'', or ``Sent 10,000 won to John.''.
  For the first case, our system should find out who is `you'
  and reduce the debt.

  \item \textbf{User want to view the debt summary} \\
  In this case user should access to the web page.
  The page should show the summary of debt/receivables,
  and detailed information of each transactions,
  such as date, amount, item, and the person.
\end{enumerate}

\section{Requirements}
\begin{enumerate}
  \item The system should know whether the message is about money or not. 
  \item The system should extract the amount of money in the message,
  and calculate for each person.
  \item The system should notice which word is the people's name,
  and assign proper money to each person mentioned in the message.
  \item The system must cover Slack, transactions database, and the webpage.
  To achieve this, the system must have an ability to send/receive HTTP request.
  Database must be set up to provide creditable records.
\end{enumerate}

\end{document}