\documentclass{article}
\usepackage[utf8]{inputenc}
\usepackage{kotex}
\usepackage{lmodern}
\usepackage{cite}

\author{지준섭}
\title{Textual Entailment-Acceptability Corpus Proposal}

\begin{document}
  
\maketitle

\section{필요성}
기존에 Textual Entailment Corpus로서 RTE, SNLI\cite{snli:emnlp2015} 등이 존재하고,
텍스트 장르의 다양성 확보를 위해 multiNLI\cite{multinli}가 기획되었으나,
News Editorials 도메인을 포함한 일정 규모 이상의 코퍼스는 존재하지 않습니다.
Textual Entailment Problem의 경우 도메인에 대해 일정 이상의 의존도를 가지고 있기에,
해당 도메인의 코퍼스를 구축할 수 있다면 Textual Entailment 관련 연구에
많은 도움이 될 것으로 생각됩니다.
또한, 현재 진행하는 Credibility Prediction Using Relations with Sentences 연구의
첫 번째 단계(Sentence Selection)에도 큰 도움이 될 수 있을 것으로 판단됩니다.

\section{목적}
기존의 코퍼스의 텍스트(영문 사설 뉴스)를 기반으로, 임의의 두 문장 간의 관계를
Entailment/Neutral/Contradiction 세 가지로 나누어 표현하고,
각각의 문장에 대해 acceptability를 채점한 코퍼스를 구축합니다.
학습을 통해 의미있는 경향성을 찾아낼 수 있는 것을 목표로 하며,
관계 분류 및 수용성 점수에 대해 라벨이 비교적 고르게 분포하는 것을 지향합니다.

\section{의의}

\section{이론적 배경}
\begin{enumerate}
  \item \textbf{multiNLI}
    10개의 장르(fiction / government / slate / telephone / travel
    / 9.11 / face-to-face / letters / oup / verbatim)에 대한
    textual entailment corpus입니다. OANC(Open American National Corpus)를
    기반으로 \underline{hypothesis를 크라우드소싱으로 모으고},
    각각의 premise \& hypothesis pair에
    entailment / neutral / contradiction의 세 라벨을 부여하였습니다.
    이 중 5개의 장르는 test set에만 포함되어,
    장르에 구애받지 않는 generality를 가짐을 증명하였습니다.
  \item \textbf{A Corpus of Sentence-level Annotations of
    Local Acceptability with Reasons}
    News Editorials 도메인에 대하여 
    (Knowledege Awareness / Verifiability / Disputatbility / Acceptance as True)
    4개 지표에 대한 응답을 기록, 데이터셋을 구축하였습니다.
\end{enumerate}

\section{가설의 설정}
두 문장 사이의 entailment 관계와
두 문장의 acceptability 점수에는 상관 관계가 존재하며,
문장 A, B 사이의 entailment 관계와 A의 acceptability 점수를 통해
B의 acceptability 점수를 예측할 수 있을 것입니다.

\section{연구 방법}
크게 두 가지의 방법을 기획하였습니다.
\begin{enumerate}
  \item multiNL\cite{multinli}의 Corpus를 기반으로 하는 연구입니다.
  multiNLI에서 이미 확보한 문장들에 대하여,
  크라우드 소싱을 통해 acceptability를 조사합니다.
  참여자들에게는 랜덤으로 한 문장씩을 보여주고,
  이전 스타랩 연구\cite{credcorpus}와 동일한 기준을 적용하여 acceptability를 매깁니다.
  한 단계의 설문으로 corpus를 구축할 수 있다는 장점이 있지만,
  multiNLI의 domain 장르에 news가 포함되지 않으므로,
  참여자들이 acceptability를 얼마나 일관성 있게 판단할 수 있을지가 가장 큰 문제가 됩니다.

  \item Reuters-21578\cite{Reuters21578} 등의 News Corpus를 기반으로 하는 연구입니다.
  총 두 번의 설문을 필요로 하며,
  먼저 참여자들에게 무작위의 document를 보여 주고, document 내의 임의의 문장에 대하여
  그 문장과 entailment / neutral / contradiction 관계에 있는 문장을
  같은 document 내에서 선택하도록 합니다.
  이후 두 번째 설문에서, 첫 번째 방식과 동일하게
  임의의 한 문장씩을 보여주고 acceptability를 판별하게 합니다.
  이 경우, 각각의 설문에 참여하는 annotator는 겹치지 않는 것이 좋을 것 같습니다.
  News 도메인에 대하여 acceptability를 측정할 수 있다는 장점이 있지만,
  두 번의 크라우드소싱 과정을 거쳐야 한다는 단점이 존재합니다.
\end{enumerate}

\section{기대되는 결과}
Corpus의 제작을 통해 두 값 사이의 관계성을 찾고,
나아가 임의의 문장 A, B 사이의 entailment 관계 / 문장 성분 / 문장 A의 acceptability 점수를 이용하여
문장 B의 acceptability를 예측하는 baseline 모델 또한 만들 수 있기를 희망합니다.
이 때, entailment관계를 gold label 뿐만이 아니라
학습된 모델의 예측 값으로 사용하는 variation도 존재할 수 있을 것입니다.

\bibliographystyle{plain}
\bibliography{arg-corpus-proposal}

\end{document}