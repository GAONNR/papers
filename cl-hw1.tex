\documentclass{article}
\usepackage[utf8]{inputenc}
\usepackage{kotex}
\usepackage{lmodern}
\usepackage{cite}
\usepackage{minted}

\author{20193601 지준섭}
\title{CS579 Homework\#1}

\begin{document}

\maketitle

\section{Chapter 3}
\textbf{Problem}: Extend the predicate travel/3 so that
it not only tells you the route to take to get from one place to another,
but also how you have to travel.
That is, the new program should let us know,
for each stage of the voyage,
whether we need to travel by car, train, or plane. \\

\noindent
\textbf{Solution}:
\begin{minted}[breaklines]{prolog}
travel(X, Y, go(X, Y, car)) :- byCar(X, Y).
travel(X, Y, go(X, Y, train)) :- byTrain(X, Y).
travel(X, Y, go(X, Y, plane)) :- byPlane(X, Y).
travel(X, Y, go(X, Z, car, Path)) :- byCar(X, Z), travel(Z, Y, Path).
travel(X, Y, go(X, Z, train, Path)) :- byTrain(X, Z), travel(Z, Y, Path).
travel(X, Y, go(X, Z, plane, Path)) :- byPlane(X, Z), travel(Z, Y, Path).
\end{minted}

\noindent
We can find the path with the command \mintinline{prolog}{travel(X, Y, Path)}.
For example:

\begin{minted}[breaklines]{prolog}
?- travel(valmont, singapore, Path).
Path = go(valmont, saarbruecken, car, go(saarbruecken, frankfurt, train, go(frankfurt, singapore, plane))) ;
Path = go(valmont, metz, car, go(metz, frankfurt, train, go(frankfurt, singapore, plane))) ;
\end{minted}

\noindent
It means there are two paths from \textit{valmont} to \textit{singapore}:
\begin{enumerate}
  \item \textit{valmont} to \textit{saarbruecken} by car,
    to \textit{frankfurt} by train, to \textit{singapore} by plane.
  \item \textit{valmont} to \textit{metz} by car,
    to \textit{frankfurt} by train, to \textit{singapore} by plane.
\end{enumerate}

\section{Chapter 4}
\textbf{Problem}: write a 3-place predicate combine3
which takes three lists as arguments and
combines the elements of the first two lists into the third as follows:

\begin{minted}[breaklines]{prolog}
?- combine3([a,b,c],[1,2,3],X).
X = [j(a,1),j(b,2),j(c,3)]
?- combine3([f,b,yip,yup],[glu,gla,gli,glo],R).
R = [j(f,glu),j(b,gla),j(yip,gli),j(yup,glo)]
\end{minted}

\noindent
\textbf{Solution}:
\begin{minted}[breaklines]{prolog}
combine3([], [], []).
combine3([Xh|Xt], [Yh|Yt], [j(Xh, Yh)|Zt]) :- combine3(Xt, Yt, Zt).
\end{minted}

\noindent
And here is the output:
\begin{minted}[breaklines]{prolog}
?- combine3([a, b, c], [d, e, f], Z).
Z = [j(a, d), j(b, e), j(c, f)].
\end{minted}


\section{Chapter 5}
\textbf{Problem}: Write a 3-place predicate dot
whose first argument is a list of integers,
whose second argument is a list of integers
of the same length as the first,
and whose third argument is the dot product of the first argument
with the second. \\

\noindent
\textbf{Solution}: Pop single number from each vector.
Accumulate the value of multiplication.
\begin{minted}[breaklines]{prolog}
dot([], [], 0).
dot([Xh|Xt], [Yh|Yt], Z) :- dot(Xt, Yt, Zz), Z is (Zz + (Xh * Yh)).
\end{minted}

\noindent
And here is the output:
\begin{minted}[breaklines]{prolog}
?- dot([2, 5, 6], [3, 4, 1], Result).
Result = 32.  
\end{minted}

\section{Chapter 6}
\textbf{Problem}: Write a predicate flatten(List,Flat)
that holds when the first argument List flattens to the second argument Flat.
This should be done without making use of append/3. \\

\noindent
\textbf{Solution}:
We use an accumulator to achieve the predicate
without using \textbf{append} method.
Head of the first argument can be an atomic element or a list,
so both head and tail should be flattened recursively. 
\begin{minted}[breaklines]{prolog}
flatten(X, Y) :- accFlatten(X, [], Y).

accFlatten([], Y, Y).
accFlatten([Xh|Xt], Y, Z) :- accFlatten(Xh, A, Z), accFlatten(Xt, Y, A).
accFlatten(X, Y, [X|Y]) :- [_|_] \= X, [] \= X.
\end{minted}

And here is the output:
\begin{minted}[breaklines]{prolog}
?- flatten([[1, 2, 3], 4, [5], 6, 7], F).
F = [1, 2, 3, 4, 5, 6, 7] ;
\end{minted}
\end{document}